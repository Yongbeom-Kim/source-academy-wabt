\subsection{Background}

The main motivation behind this document is to provide a specification for the WebAssembly Text
format used in the Source Academy. \vspace{1em}

This differs from the official WebAssembly specifications
in that this document is meant to be a specification of the WebAssembly Text features implemented
in Source Academy, and that it is also meant for users to write and understand WebAssembly Text as
a language, rather than to provide a industry-wide specification for WebAssembly as a whole,
including runtime, binary and verification details among others.  \vspace{1em}

In short, this document is meant to be a specification for the WebAssembly Text format supported and used
in the Source Academy (or the \textbf{\texttt{wasm}} module).

\subsection{About WebAssembly Text}
WebAssembly Text is a text format for WebAssembly modules. The design of the 
WebAssembly runtime and instruction set are beyond the scope of this specification, 
and can be read in the official WebAssembly 
\href{https://webassembly.github.io/spec/core/_download/WebAssembly.pdf}{specification}. \vspace{1em}

Notably, the computational model of WebAssembly is based on a stack machine, 
where a sequence of instructions are executed in order. Instructions consume 
values on an implicit operand stack, and push any results back onto the stack.
The WebAssembly Text format is a rendering of the above syntax into S-expressions.

\subsection{Differences between official WebAssembly Text Format}

Here are documented differences between the current specifications and the official WebAssembly Text specifications.

\subsubsection{Data \& Data Count Segment}

The data and data count segments in the official WebAssembly Text specifications are omitted and not support in the current iteration of the WebAssembly Text compiler.
Since the data segment is used to initialize the memory segment, users cannot currently initialise the heap memory with a pre-set array of values.