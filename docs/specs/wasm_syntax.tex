The following rules concern basic WebAssembly Text syntax.

\subsection{White Space}
White space is any sequence of the following: space (U+020), horizontal tab (U+09), line feed (U+0A), carriage return (U+0D) or comments. White space is ignored except as it separates tokens that would otherwise combine into a single token.

\begin{alignat*}{9}
    && \textit{space}    &&\quad ::= &\quad && (\textit{U+20}\ |\ \textit{format}\ |\ \textit{comment})^*\ \quad && \\[1mm]
    && \textit{format}    &&\quad ::= &\quad && \textit{newline}\  \quad && \\[1mm]
    &&    &&\quad | &\quad && \textit{U+09}  \quad && \textrm{horizontal tab} \\[1mm]
    && \textit{newline}    &&\quad ::= &\quad && \textit{U+0A}\  \quad && \textrm{line feed} \\[1mm]
    &&    &&\quad | &\quad && \textit{U+0D}\ \quad && \textrm{carriage return} \\[1mm]
    &&    &&\quad | &\quad && \textit{U+0D U+0A}\  \quad && \textrm{line feed + carriage return} \\[1mm]
\end{alignat*}

A line comment starts with a double semicolon (\textbf{\texttt{;;}}) and continues to the end of the line, whereas a block comment is enclosed in parentheses and semicolons (\textbf{\texttt{(;}} and \textbf{\texttt{;)}}). Block comments can be nested.

\begin{alignat*}{9}
    && \textit{comment}    &&\quad ::= &\quad && \textit{linecomment}\ |\ \textit{blockcomment}  \quad && \\[1mm]
    && \textit{linecomment}    &&\quad ::= &\quad && \textbf{\texttt{;;}}\ \textit{char}^*\ (\textit{newline}\ |\ \textit{eof})  \quad && \\[1mm]
    && \textit{blockcomment}    &&\quad ::= &\quad && \textbf{\texttt{(;}}\ \textit{blockchar}^*\ \textbf{\texttt{;)}}\  \quad && \\[1mm]
    && \textit{blockchar}&&\quad ::= &\quad && \textit{char} && \textrm{if char is not ; or (} \\[1mm]
    && &&\quad | &\quad && \textrm{\texttt{;}} && \textrm{if next char is not )} \\[1mm]
    && &&\quad | &\quad && \textrm{\texttt{(}} && \textrm{if next char is not ;} \\[1mm]
    && &&\quad | &\quad && \textit{blockcomment} && \\[1mm]
\end{alignat*}

\subsection{Strings}
A string is a sequence of characters encoded as UTF-8. A string must be enclosed in quotation marks and may contain any character other than ASCII44 control characters, quotation marks ("), or backslash (\textbackslash), except when expressed with an escape sequence.

\begin{alignat*}{9}
    && \textit{string}    &&\quad ::= &\quad && \textbf{\texttt{"}}\ \textit{stringelem}\ \textbf{\texttt{"}}\  \quad && \\[1mm]
    && \textit{stringelem}    &&\quad ::= &\quad && \textinterrobangdown{stringchar}  \quad && \\[1mm]
    && \textit{stringelem}    &&\quad ::= &\quad && \textbf{\texttt{\textbackslash}}\ \textit{hexdigit}\ \textit{hexdigit}\  \quad && \textrm{2-digit hexcode} \\[1mm]
    && \textit{module-name}    &&\quad ::= &\quad && \textit{string} &&\\[1mm]
    && \textit{name}    &&\quad ::= &\quad && \textit{string} &&\\[1mm]
    && \textit{import-desc}       &&\quad ::= &\quad && \textbf{\texttt{(func}}\ \textit{id}^?\ \textit{typeuse}\ \textbf{\texttt{)}}  \quad && \textrm{function import} \\[1mm]
    &&                            &&\quad  |  &\quad && \textbf{\texttt{(table}}\ \textit{id}^?\ \textit{tabletype}\ \textbf{\texttt{)}}  \quad && \textrm{table import} \\[1mm]
    &&                            &&\quad  |  &\quad && \textbf{\texttt{(memory}}\ \textit{id}^?\ \textit{memtype}\ \textbf{\texttt{)}}  \quad && \textrm{memory import} \\[1mm]
    &&                            &&\quad  |  &\quad && \textbf{\texttt{(global}}\ \textit{id}^?\ \textit{globaltype}\ \textbf{\texttt{)}}  \quad && \textrm{global import} \\[1mm]
\end{alignat*}

\subsection{Names}
A name is string.

\begin{alignat*}{9}
    && \textit{name}    &&\quad ::= &\quad && \textit{string}  \quad && \\[1mm]
\end{alignat*}

\subsection{Identifiers}

Each definition can be identified by either its index or symbolic identifier. Symbolic identifiers are identifiers that start with a dollar sign (\$) followed by a name. A name is a string that does not contain a space, quotation mark, comma, semicolon or bracket.

\begin{alignat*}{9}
    && \textit{id}    &&\quad ::= &\quad && \textbf{\texttt{\$}}\ \textit{idchar}^+  \quad && \\[1mm]
    && \textit{idchar}    &&\quad ::= &\quad && \textbf{\texttt{0}}\ |\ \textbf{\texttt{\dots}}\ |\ \textbf{\texttt{9}}\ |  \quad && \\[1mm]
    &&                    &&\quad  |  &\quad && \textbf{\texttt{a}}\ |\ \textbf{\texttt{\dots}}\ |\ \textbf{\texttt{z}}\ |  \quad && \\[1mm]
    &&                    &&\quad  |  &\quad && \textbf{\texttt{A}}\ |\ \textbf{\texttt{\dots}}\ |\ \textbf{\texttt{Z}}\ |  \quad && \\[1mm]
    &&                    &&\quad  |  &\quad && \textbf{\texttt{!}}\ |\ \textbf{\texttt{\#}}\ |\ \textbf{\texttt{\$}}\ |\ \textbf{\texttt{\%}}\ |\ \textbf{\texttt{\&}}\ |\ \textbf{\texttt{*}}\ |\ \textbf{\texttt{+}}\ |\ \textbf{\texttt{-}}\ |\ \textbf{\texttt{.}}\ |\ \textbf{\texttt{/}}\ |\ \textbf{\texttt{:}}\ |\ \textbf{\texttt{<}}\  \quad && \\[1mm]
    &&                    &&\quad  |  &\quad && \textbf{\texttt{=}}\ |\ \textbf{\texttt{>}}\ |\ \textbf{\texttt{?}}\ |\ \textbf{\texttt{@}}\ |\ \textbf{\texttt{\textbackslash}}\ |\ \textbf{\texttt{\^}}\ |\ \textbf{\texttt{\_}}\ |\ \textbf{\texttt{`}}\ |\ \textbf{\texttt{|}}\ |\ \textbf{\texttt{~}}\  \quad && \\[1mm]
\end{alignat*}

\subsection{Types}
The following are the available types in WebAssembly.
\subsubsection{Number Types}
All numbers are either 32- or 64-bit integers or floating points.

\begin{alignat*}{9}
    && \textit{numtype}    &&\quad ::= &\quad && \textbf{\texttt{i32}}  \quad && \\[1mm]
    && &&\quad | &\quad && \textbf{\texttt{i64}}  \quad && \\[1mm]
    && &&\quad | &\quad && \textbf{\texttt{f32}}  \quad && \\[1mm]
    && &&\quad | &\quad && \textbf{\texttt{f64}}  \quad && \\[1mm]
\end{alignat*}

\subsubsection{Reference Types}
Reference types are first-class references to objects. 
\textbf{\texttt{funcref}} is a reference to a function. \textbf{\texttt{externref}} is a reference to an external object.
\begin{alignat*}{9}
    && \textit{reftype}    &&\quad ::= &\quad && \textbf{\texttt{funcref}}  \quad && \\[1mm]
    && &&\quad | &\quad && \textbf{\texttt{externref}}  \quad && \\[1mm]
    && \textit{heaptype}    &&\quad ::= &\quad && \textbf{\texttt{func}}  \quad && \\[1mm]
    && &&\quad | &\quad && \textbf{\texttt{extern}}  \quad && \\[1mm]
\end{alignat*}

\subsubsection{Value Types}
\begin{alignat*}{9}
    && \textit{valtype}    &&\quad ::= &\quad && \textbf{\texttt{numtype}}  \quad && \\[1mm]
    &&     &&\quad | &\quad && \textbf{\texttt{reftype}}  \quad && \\[1mm]
\end{alignat*}

\subsubsection{Function Types}
The type of a function is determined by its parameters and result, where the function maps the parameter types to the result types. The parameters and result are value types.

\begin{alignat*}{9}
    && \textit{functype}    &&\quad ::= &\quad && \textbf{\texttt{(}}\ \textbf{\texttt{func}}\ \textit{param}^*\ \textit{result}^*\ \textbf{\texttt{)}}\  \quad && \\[1mm]
    && \textit{param}    &&\quad ::= &\quad && \textbf{\texttt{(}}\ \textbf{\texttt{param}}\ \textit{id}^?\ \textit{valtype}\ \textbf{\texttt{)}}\  \quad && \\[1mm]
    && \textit{result}    &&\quad ::= &\quad && \textbf{\texttt{(}}\ \textbf{\texttt{result}}\ \textit{valtype}\ \textbf{\texttt{)}}\  \quad && \\[1mm]
\end{alignat*}

\subsubsection{Global Types}
\textit{Globals} refer to global variables. A global type is a value type and a mutability flag.
\begin{alignat*}{9}
    && \textit{globaltype}    &&\quad ::= &\quad && \textit{mut}\ \textit{valtype}  \quad && \\[1mm]
    && \textit{mut} &&\quad | &\quad && \textit{const}\ |\ \textit{var} \quad && \\[1mm]
\end{alignat*}
